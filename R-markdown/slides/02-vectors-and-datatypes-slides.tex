\PassOptionsToPackage{unicode=true}{hyperref} % options for packages loaded elsewhere
\PassOptionsToPackage{hyphens}{url}
\PassOptionsToPackage{dvipsnames,svgnames*,x11names*}{xcolor}
%
\documentclass[
  11pt,
  ignorenonframetext,
]{beamer}
\usepackage{pgfpages}
\setbeamertemplate{caption}[numbered]
\setbeamertemplate{caption label separator}{: }
\setbeamercolor{caption name}{fg=normal text.fg}
\beamertemplatenavigationsymbolsempty
% Prevent slide breaks in the middle of a paragraph:
\widowpenalties 1 10000
\raggedbottom
\setbeamertemplate{part page}{
  \centering
  \begin{beamercolorbox}[sep=16pt,center]{part title}
    \usebeamerfont{part title}\insertpart\par
  \end{beamercolorbox}
}
\setbeamertemplate{section page}{
  \centering
  \begin{beamercolorbox}[sep=12pt,center]{part title}
    \usebeamerfont{section title}\insertsection\par
  \end{beamercolorbox}
}
\setbeamertemplate{subsection page}{
  \centering
  \begin{beamercolorbox}[sep=8pt,center]{part title}
    \usebeamerfont{subsection title}\insertsubsection\par
  \end{beamercolorbox}
}
\AtBeginPart{
  \frame{\partpage}
}
\AtBeginSection{
  \ifbibliography
  \else
    \frame{\sectionpage}
  \fi
}
\AtBeginSubsection{
  \frame{\subsectionpage}
}
\usepackage{lmodern}
\usepackage{amssymb,amsmath}
\usepackage{ifxetex,ifluatex}
\ifnum 0\ifxetex 1\fi\ifluatex 1\fi=0 % if pdftex
  \usepackage[T1]{fontenc}
  \usepackage[utf8]{inputenc}
  \usepackage{textcomp} % provides euro and other symbols
\else % if luatex or xelatex
  \usepackage{unicode-math}
  \defaultfontfeatures{Scale=MatchLowercase}
  \defaultfontfeatures[\rmfamily]{Ligatures=TeX,Scale=1}
\fi
% use upquote if available, for straight quotes in verbatim environments
\IfFileExists{upquote.sty}{\usepackage{upquote}}{}
\IfFileExists{microtype.sty}{% use microtype if available
  \usepackage[]{microtype}
  \UseMicrotypeSet[protrusion]{basicmath} % disable protrusion for tt fonts
}{}
\makeatletter
\@ifundefined{KOMAClassName}{% if non-KOMA class
  \IfFileExists{parskip.sty}{%
    \usepackage{parskip}
  }{% else
    \setlength{\parindent}{0pt}
    \setlength{\parskip}{6pt plus 2pt minus 1pt}}
}{% if KOMA class
  \KOMAoptions{parskip=half}}
\makeatother
\usepackage{xcolor}
\IfFileExists{xurl.sty}{\usepackage{xurl}}{} % add URL line breaks if available
\IfFileExists{bookmark.sty}{\usepackage{bookmark}}{\usepackage{hyperref}}
\hypersetup{
  pdftitle={Introduction to R},
  pdfauthor={UA R Users Group},
  colorlinks=true,
  linkcolor=Maroon,
  filecolor=Maroon,
  citecolor=Blue,
  urlcolor=blue,
  breaklinks=true}
\urlstyle{same}  % don't use monospace font for urls
\usepackage[margin=1in]{geometry}
\newif\ifbibliography
\usepackage{color}
\usepackage{fancyvrb}
\newcommand{\VerbBar}{|}
\newcommand{\VERB}{\Verb[commandchars=\\\{\}]}
\DefineVerbatimEnvironment{Highlighting}{Verbatim}{commandchars=\\\{\}}
% Add ',fontsize=\small' for more characters per line
\usepackage{framed}
\definecolor{shadecolor}{RGB}{248,248,248}
\newenvironment{Shaded}{\begin{snugshade}}{\end{snugshade}}
\newcommand{\AlertTok}[1]{\textcolor[rgb]{0.94,0.16,0.16}{#1}}
\newcommand{\AnnotationTok}[1]{\textcolor[rgb]{0.56,0.35,0.01}{\textbf{\textit{#1}}}}
\newcommand{\AttributeTok}[1]{\textcolor[rgb]{0.77,0.63,0.00}{#1}}
\newcommand{\BaseNTok}[1]{\textcolor[rgb]{0.00,0.00,0.81}{#1}}
\newcommand{\BuiltInTok}[1]{#1}
\newcommand{\CharTok}[1]{\textcolor[rgb]{0.31,0.60,0.02}{#1}}
\newcommand{\CommentTok}[1]{\textcolor[rgb]{0.56,0.35,0.01}{\textit{#1}}}
\newcommand{\CommentVarTok}[1]{\textcolor[rgb]{0.56,0.35,0.01}{\textbf{\textit{#1}}}}
\newcommand{\ConstantTok}[1]{\textcolor[rgb]{0.00,0.00,0.00}{#1}}
\newcommand{\ControlFlowTok}[1]{\textcolor[rgb]{0.13,0.29,0.53}{\textbf{#1}}}
\newcommand{\DataTypeTok}[1]{\textcolor[rgb]{0.13,0.29,0.53}{#1}}
\newcommand{\DecValTok}[1]{\textcolor[rgb]{0.00,0.00,0.81}{#1}}
\newcommand{\DocumentationTok}[1]{\textcolor[rgb]{0.56,0.35,0.01}{\textbf{\textit{#1}}}}
\newcommand{\ErrorTok}[1]{\textcolor[rgb]{0.64,0.00,0.00}{\textbf{#1}}}
\newcommand{\ExtensionTok}[1]{#1}
\newcommand{\FloatTok}[1]{\textcolor[rgb]{0.00,0.00,0.81}{#1}}
\newcommand{\FunctionTok}[1]{\textcolor[rgb]{0.00,0.00,0.00}{#1}}
\newcommand{\ImportTok}[1]{#1}
\newcommand{\InformationTok}[1]{\textcolor[rgb]{0.56,0.35,0.01}{\textbf{\textit{#1}}}}
\newcommand{\KeywordTok}[1]{\textcolor[rgb]{0.13,0.29,0.53}{\textbf{#1}}}
\newcommand{\NormalTok}[1]{#1}
\newcommand{\OperatorTok}[1]{\textcolor[rgb]{0.81,0.36,0.00}{\textbf{#1}}}
\newcommand{\OtherTok}[1]{\textcolor[rgb]{0.56,0.35,0.01}{#1}}
\newcommand{\PreprocessorTok}[1]{\textcolor[rgb]{0.56,0.35,0.01}{\textit{#1}}}
\newcommand{\RegionMarkerTok}[1]{#1}
\newcommand{\SpecialCharTok}[1]{\textcolor[rgb]{0.00,0.00,0.00}{#1}}
\newcommand{\SpecialStringTok}[1]{\textcolor[rgb]{0.31,0.60,0.02}{#1}}
\newcommand{\StringTok}[1]{\textcolor[rgb]{0.31,0.60,0.02}{#1}}
\newcommand{\VariableTok}[1]{\textcolor[rgb]{0.00,0.00,0.00}{#1}}
\newcommand{\VerbatimStringTok}[1]{\textcolor[rgb]{0.31,0.60,0.02}{#1}}
\newcommand{\WarningTok}[1]{\textcolor[rgb]{0.56,0.35,0.01}{\textbf{\textit{#1}}}}
\setlength{\emergencystretch}{3em}  % prevent overfull lines
\providecommand{\tightlist}{%
  \setlength{\itemsep}{0pt}\setlength{\parskip}{0pt}}
\setcounter{secnumdepth}{-2}

% set default figure placement to htbp
\makeatletter
\def\fps@figure{htbp}
\makeatother


\title{Introduction to R}
\author{UA R Users Group}
\date{May 14, 2019}

\begin{document}
\frame{\titlepage}

\begin{frame}

\begin{quote}
\hypertarget{learning-objectives}{%
\subsection{Learning Objectives}\label{learning-objectives}}

\begin{itemize}
\tightlist
\item
  Familiarize participants with R syntax
\item
  Understand the concepts of objects and assignment
\item
  Understand the concepts of vectors and data types
\item
  Get exposed to a few functions
\end{itemize}
\end{quote}

\end{frame}

\begin{frame}{The R syntax}
\protect\hypertarget{the-r-syntax}{}

\end{frame}

\begin{frame}[fragile]{Creating objects}
\protect\hypertarget{creating-objects}{}

You can get output from R simply by typing in math in the console

\begin{Shaded}
\begin{Highlighting}[]
\DecValTok{3} \OperatorTok{+}\StringTok{ }\DecValTok{5}
\end{Highlighting}
\end{Shaded}

\begin{verbatim}
## [1] 8
\end{verbatim}

\begin{Shaded}
\begin{Highlighting}[]
\DecValTok{12}\OperatorTok{/}\DecValTok{7}
\end{Highlighting}
\end{Shaded}

\begin{verbatim}
## [1] 1.714286
\end{verbatim}

\begin{Shaded}
\begin{Highlighting}[]
\DecValTok{2}\OperatorTok{*}\DecValTok{4}
\end{Highlighting}
\end{Shaded}

\begin{verbatim}
## [1] 8
\end{verbatim}

\begin{Shaded}
\begin{Highlighting}[]
\DecValTok{2}\OperatorTok{^}\DecValTok{4}
\end{Highlighting}
\end{Shaded}

\begin{verbatim}
## [1] 16
\end{verbatim}

We can also comment what it is we're doing

\begin{Shaded}
\begin{Highlighting}[]
\CommentTok{# I am adding 3 and 5. R is fun!}
\DecValTok{3} \OperatorTok{+}\StringTok{ }\DecValTok{5}
\end{Highlighting}
\end{Shaded}

\begin{verbatim}
## [1] 8
\end{verbatim}

What happens if we do that same command without the \# sign in the
front?

\begin{Shaded}
\begin{Highlighting}[]
\NormalTok{I am adding }\DecValTok{3}\NormalTok{ and }\FloatTok{5.}\NormalTok{ R is fun}\OperatorTok{!}
\DecValTok{3} \OperatorTok{+}\StringTok{ }\DecValTok{5}
\end{Highlighting}
\end{Shaded}

Now R is trying to run that sentence as a command, and it doesn't work.
Now we're stuck over in the console. The \texttt{+} sign means that it's
still waiting for input, so we can't type in a new command. To get out
of this type \texttt{Esc}. This will work whenever you're stuck with
that \texttt{+} sign.

It's great that R is a glorified caluculator, but obviously we want to
do more interesting things.

To do useful and interesting things, we need to assign \emph{values} to
\emph{objects}. To create objects, we need to give it a name followed by
the assignment operator \texttt{\textless{}-} and the value we want to
give it.

\end{frame}

\begin{frame}[fragile]{Assignment operator}
\protect\hypertarget{assignment-operator}{}

For instance, instead of adding 3 + 5, we can assign those values to
objects and then add them.

\begin{Shaded}
\begin{Highlighting}[]
\CommentTok{# assign 3 to a}
\NormalTok{a <-}\StringTok{ }\DecValTok{3}
\CommentTok{# assign 5 to b}
\NormalTok{b <-}\StringTok{ }\DecValTok{5}
\end{Highlighting}
\end{Shaded}

\begin{Shaded}
\begin{Highlighting}[]
\CommentTok{# what now is a}
\NormalTok{a}
\end{Highlighting}
\end{Shaded}

\begin{verbatim}
## [1] 3
\end{verbatim}

\begin{Shaded}
\begin{Highlighting}[]
\CommentTok{# what now is b}
\NormalTok{b}
\end{Highlighting}
\end{Shaded}

\begin{verbatim}
## [1] 5
\end{verbatim}

\begin{Shaded}
\begin{Highlighting}[]
\CommentTok{# Add a and b}
\NormalTok{a }\OperatorTok{+}\StringTok{ }\NormalTok{b}
\end{Highlighting}
\end{Shaded}

\begin{verbatim}
## [1] 8
\end{verbatim}

\texttt{\textless{}-} is the assignment operator. It assigns values on
the right to objects on the left. So, after executing
\texttt{x\ \textless{}-\ 3}, the value of \texttt{x} is \texttt{3}. The
arrow can be read as 3 \textbf{goes into} \texttt{x}. You can also use
\texttt{=} for assignments but not in all contexts so it is good
practice to use \texttt{\textless{}-} for assignments. \texttt{=} should
only be used to specify the values of arguments in functions, see below.

In RStudio, typing \texttt{Alt\ +\ -} (push \texttt{Alt}, the key next
to your space bar at the same time as the \texttt{-} key) will write
\texttt{\textless{}-} in a single keystroke.

To view which objects we have stored in memory, we can use the
\texttt{ls()} command

\begin{Shaded}
\begin{Highlighting}[]
\KeywordTok{ls}\NormalTok{()}
\end{Highlighting}
\end{Shaded}

\begin{verbatim}
## [1] "a" "b"
\end{verbatim}

To remove objects we can use the \texttt{rm()} command

\begin{Shaded}
\begin{Highlighting}[]
\KeywordTok{rm}\NormalTok{(a)}
\end{Highlighting}
\end{Shaded}

\begin{block}{Exercise}

\begin{itemize}
\tightlist
\item
  What happens if we change \texttt{a} and then re-add \texttt{a} and
  \texttt{b}?
\item
  Does it work if you just change \texttt{a} in the script and then add
  \texttt{a} and \texttt{b}? Did you still get the same answer after
  they changed \texttt{a}? If so, why do you think that might be?
\item
  We can also assign a + b to a new variable, \texttt{c}. How would you
  do this?
\end{itemize}

\end{block}

\end{frame}

\begin{frame}[fragile]{Notes on objects}
\protect\hypertarget{notes-on-objects}{}

Objects can be given any name such as \texttt{x},
\texttt{current\_temperature}, or \texttt{subject\_id}. You want your
object names to be explicit and not too long. They cannot start with a
number (\texttt{2x} is not valid but \texttt{x2} is). R is case
sensitive (e.g., \texttt{my\_data} is different from \texttt{My\_data}).
There are some names that cannot be used because they represent the
names of fundamental functions in R (e.g., \texttt{if}, \texttt{else},
\texttt{for}, see
\href{https://stat.ethz.ch/R-manual/R-devel/library/base/html/Reserved.html}{here}
for a complete list). In general, even if it's allowed, it's best to not
use other function names (e.g., \texttt{c}, \texttt{T}, \texttt{mean},
\texttt{data}, \texttt{df}, \texttt{weights}). In doubt check the help
to see if the name is already in use. It's also best to avoid dots
(\texttt{.}) within a variable name as in \texttt{my.dataset}. There are
many functions in R with dots in their names for historical reasons, but
because dots have a special meaning in R (for methods) and other
programming languages, it's best to avoid them. It is also recommended
to use nouns for variable names, and verbs for function names. It's
important to be consistent in the styling of your code (where you put
spaces, how you name variable, etc.). In R, two popular style guides are
\href{http://adv-r.had.co.nz/Style.html}{Hadley Wickham's} and
\href{https://google-styleguide.googlecode.com/svn/trunk/Rguide.xml}{Google's}.

When assigning a value to an object, R does not print anything. You can
force to print the value by using parentheses or by typing the name:

\end{frame}

\begin{frame}[fragile]{Functions}
\protect\hypertarget{functions}{}

The other key feature of R are functions. These are R's built in
capabilities. Some examples of these are mathematical functions, like
\texttt{sqrt} and \texttt{round}. You can also get functions from
libraries (which we'll talk about in a bit), or even write your own.

\end{frame}

\begin{frame}[fragile]{Functions and their arguments}
\protect\hypertarget{functions-and-their-arguments}{}

Functions are ``canned scripts'' that automate something complicated or
convenient or both. Many functions are predefined, or become available
when using the function \texttt{library()} (more on that later). A
function usually gets one or more inputs called \emph{arguments}.
Functions often (but not always) return a \emph{value}. A typical
example would be the function \texttt{sqrt()}. The input (the argument)
must be a number, and the return value (in fact, the output) is the
square root of that number. Executing a function (`running it') is
called \emph{calling} the function. An example of a function call is:

\texttt{sqrt(a)}

Here, the value of \texttt{a} is given to the \texttt{sqrt()} function,
the \texttt{sqrt()} function calculates the square root. This function
is very simple, because it takes just one argument.

The return `value' of a function need not be numerical (like that of
\texttt{sqrt()}), and it also does not need to be a single item: it can
be a set of things, or even a data set. We'll see that when we read data
files in to R.

Arguments can be anything, not only numbers or filenames, but also other
objects. Exactly what each argument means differs per function, and must
be looked up in the documentation (see below). If an argument alters the
way the function operates, such as whether to ignore `bad values', such
an argument is sometimes called an \emph{option}.

Most functions can take several arguments, but many have so-called
\emph{defaults}. If you don't specify such an argument when calling the
function, the function itself will fall back on using the
\emph{default}. This is a standard value that the author of the function
specified as being ``good enough in standard cases''. An example would
be what symbol to use in a plot. However, if you want something
specific, simply change the argument yourself with a value of your
choice.

Let's try a function that can take multiple arguments \texttt{round}.

\begin{Shaded}
\begin{Highlighting}[]
\KeywordTok{round}\NormalTok{(}\FloatTok{3.14159}\NormalTok{)}
\end{Highlighting}
\end{Shaded}

\begin{verbatim}
## [1] 3
\end{verbatim}

We can see that we get \texttt{3}. That's because the default is to
round to the nearest whole number. If we want more digits we can see how
to do that by getting information about the \texttt{round} function. We
can use \texttt{args(round)} or look at the help for this function using
\texttt{?round}.

\begin{Shaded}
\begin{Highlighting}[]
\KeywordTok{args}\NormalTok{(round)}
\end{Highlighting}
\end{Shaded}

\begin{verbatim}
## function (x, digits = 0) 
## NULL
\end{verbatim}

\begin{Shaded}
\begin{Highlighting}[]
\NormalTok{?round}
\end{Highlighting}
\end{Shaded}

We see that if we want a different number of digits, we can type
\texttt{digits=2} or however many we want.

\begin{Shaded}
\begin{Highlighting}[]
\KeywordTok{round}\NormalTok{(}\FloatTok{3.14159}\NormalTok{, }\DataTypeTok{digits =} \DecValTok{2}\NormalTok{)}
\end{Highlighting}
\end{Shaded}

\begin{verbatim}
## [1] 3.14
\end{verbatim}

If you provide the arguments in the exact same order as they are defined
you don't have to name them:

\begin{Shaded}
\begin{Highlighting}[]
\KeywordTok{round}\NormalTok{(}\FloatTok{3.14159}\NormalTok{, }\DecValTok{2}\NormalTok{)}
\end{Highlighting}
\end{Shaded}

\begin{verbatim}
## [1] 3.14
\end{verbatim}

However, it's usually not recommended practice because it's a lot of
remembering to do, and if you share your code with others that includes
less known functions it makes your code difficult to read. (It's however
OK to not include the names of the arguments for basic functions like
\texttt{mean}, \texttt{min}, etc\ldots{})

Another advantage of naming arguments, is that the order doesn't matter.
This is useful when there start to be more arguments.

\end{frame}

\begin{frame}[fragile]{Vectors and data types}
\protect\hypertarget{vectors-and-data-types}{}

A vector is the most common and basic data structure in R, and is pretty
much the workhorse of R. It's basically just a list of values, mainly
either numbers or characters. They're special lists that you can do math
with. You can assign this list of values to a variable, just like you
would for one item. You can add elements to your vector simply by using
the \texttt{c()} function, which stands for combine:

\begin{Shaded}
\begin{Highlighting}[]
\NormalTok{one_to_five <-}\StringTok{ }\KeywordTok{c}\NormalTok{(}\DecValTok{1}\NormalTok{, }\DecValTok{2}\NormalTok{, }\DecValTok{3}\NormalTok{, }\DecValTok{4}\NormalTok{, }\DecValTok{5}\NormalTok{)}
\NormalTok{one_to_five <-}\StringTok{ }\DecValTok{1}\OperatorTok{:}\DecValTok{5}
\NormalTok{one_to_five}
\end{Highlighting}
\end{Shaded}

\begin{verbatim}
## [1] 1 2 3 4 5
\end{verbatim}

A vector can also contain characters:

\begin{Shaded}
\begin{Highlighting}[]
\NormalTok{primary_colors <-}\StringTok{ }\KeywordTok{c}\NormalTok{(}\StringTok{"red"}\NormalTok{, }\StringTok{"yellow"}\NormalTok{, }\StringTok{"blue"}\NormalTok{)}
\NormalTok{primary_colors}
\end{Highlighting}
\end{Shaded}

\begin{verbatim}
## [1] "red"    "yellow" "blue"
\end{verbatim}

There are many functions that allow you to inspect the content of a
vector. \texttt{length()} tells you how many elements are in a
particular vector:

\begin{Shaded}
\begin{Highlighting}[]
\KeywordTok{length}\NormalTok{(one_to_five)}
\end{Highlighting}
\end{Shaded}

\begin{verbatim}
## [1] 5
\end{verbatim}

\begin{Shaded}
\begin{Highlighting}[]
\KeywordTok{length}\NormalTok{(primary_colors)}
\end{Highlighting}
\end{Shaded}

\begin{verbatim}
## [1] 3
\end{verbatim}

You can also do math with whole vectors. For instance if we wanted to
multiply all the values in our vector by a scalar, we can do

\begin{Shaded}
\begin{Highlighting}[]
\DecValTok{5} \OperatorTok{*}\StringTok{ }\NormalTok{one_to_five}
\end{Highlighting}
\end{Shaded}

\begin{verbatim}
## [1]  5 10 15 20 25
\end{verbatim}

or we can add the data in the two vectors together

\begin{Shaded}
\begin{Highlighting}[]
\NormalTok{two_to_ten <-}\StringTok{ }\NormalTok{one_to_five }\OperatorTok{+}\StringTok{ }\NormalTok{one_to_five}
\NormalTok{two_to_ten}
\end{Highlighting}
\end{Shaded}

\begin{verbatim}
## [1]  2  4  6  8 10
\end{verbatim}

This is very useful if we have data in different vectors that we want to
combine or work with.

There are few ways to figure out what's going on in a vector.

\texttt{class()} indicates the class (the type of element) of an object:

\begin{Shaded}
\begin{Highlighting}[]
\KeywordTok{class}\NormalTok{(one_to_five)}
\end{Highlighting}
\end{Shaded}

\begin{verbatim}
## [1] "integer"
\end{verbatim}

\begin{Shaded}
\begin{Highlighting}[]
\KeywordTok{class}\NormalTok{(primary_colors)}
\end{Highlighting}
\end{Shaded}

\begin{verbatim}
## [1] "character"
\end{verbatim}

\begin{Shaded}
\begin{Highlighting}[]
\NormalTok{new_digits <-}\StringTok{ }\KeywordTok{c}\NormalTok{(one_to_five, }\DecValTok{90}\NormalTok{) }\CommentTok{# adding at the end}
\NormalTok{new_digits <-}\StringTok{ }\KeywordTok{c}\NormalTok{(}\DecValTok{30}\NormalTok{, new_digits) }\CommentTok{# adding at the beginning}
\NormalTok{new_digits}
\end{Highlighting}
\end{Shaded}

\begin{verbatim}
## [1] 30  1  2  3  4  5 90
\end{verbatim}

What happens here is that we take the original vector
\texttt{one\_to\_five}, and we are adding another item first to the end
of the other ones, and then another item at the beginning. We can do
this over and over again to build a vector or a dataset. As we program,
this may be useful to autoupdate results that we are collecting or
calculating.

We just saw 2 of the \textbf{data types} that R uses:
\texttt{"character"} and \texttt{"integer"}. The others you will likely
encounter during data analysis are:

\begin{itemize}
\tightlist
\item
  \texttt{"logical"} for \texttt{TRUE} and \texttt{FALSE} (the boolean
  data type)
\item
  \texttt{"numeric"} for floating point decimal numbers\\
\item
  \texttt{"factor"} for categorical data. Similar to
  \texttt{"character"} data, but factors have levels
\end{itemize}

Importantly, a vector can only contain \textbf{one} data type. If you
combine multiple data types in a vector with the \texttt{c()} command, R
will try to coerce all the values to the same data type. If it cannot,
it will throw an error.

For example, what data type is our \texttt{one\_to\_five} vector if we
divide it by 2?

\begin{Shaded}
\begin{Highlighting}[]
\NormalTok{divided_integers <-}\StringTok{ }\NormalTok{one_to_five}\OperatorTok{/}\DecValTok{2}
\NormalTok{divided_integers}
\end{Highlighting}
\end{Shaded}

\begin{verbatim}
## [1] 0.5 1.0 1.5 2.0 2.5
\end{verbatim}

\begin{Shaded}
\begin{Highlighting}[]
\KeywordTok{class}\NormalTok{(divided_integers)}
\end{Highlighting}
\end{Shaded}

\begin{verbatim}
## [1] "numeric"
\end{verbatim}

Vectors are indexed sets, which means that every value can be referred
to by its order in the vector. R indexes start at 1. Programming
languages like Fortran, MATLAB, and R start counting at 1, because
that's what human beings typically do. Languages in the C family
(including C++, Java, Perl, and Python) count from 0 because that's
simpler for computers to do.

We can index a vector in many different ways. We can specify a position
of a single value, a range of values, or a vector of values. We can even
specify which values to remove by their indices.

\begin{Shaded}
\begin{Highlighting}[]
\NormalTok{one_to_five[}\DecValTok{3}\NormalTok{]}
\end{Highlighting}
\end{Shaded}

\begin{verbatim}
## [1] 3
\end{verbatim}

\begin{Shaded}
\begin{Highlighting}[]
\NormalTok{one_to_five[}\DecValTok{1}\OperatorTok{:}\DecValTok{3}\NormalTok{]}
\end{Highlighting}
\end{Shaded}

\begin{verbatim}
## [1] 1 2 3
\end{verbatim}

\begin{Shaded}
\begin{Highlighting}[]
\NormalTok{one_to_five[}\KeywordTok{c}\NormalTok{(}\DecValTok{1}\NormalTok{, }\DecValTok{3}\NormalTok{, }\DecValTok{5}\NormalTok{)]}
\end{Highlighting}
\end{Shaded}

\begin{verbatim}
## [1] 1 3 5
\end{verbatim}

\begin{Shaded}
\begin{Highlighting}[]
\NormalTok{one_to_five[}\OperatorTok{-}\DecValTok{2}\NormalTok{]}
\end{Highlighting}
\end{Shaded}

\begin{verbatim}
## [1] 1 3 4 5
\end{verbatim}

\end{frame}

\begin{frame}[fragile]{Other data structures}
\protect\hypertarget{other-data-structures}{}

Vectors are one of the many \textbf{data structures} that R uses. Other
important ones are lists (\texttt{list}), matrices (\texttt{matrix}),
and data frames (\texttt{data.frame})

\end{frame}

\end{document}
